\ifx\allfiles\undefined
\documentclass[11 pt,t]{beamer}
\geometry{showframe,paperwidth=160mm,paperheight=120mm,margin=5mm,nohead,nofoot,nomarginpar}
%\usetheme[
%bullet=circle,		% Other option: square
%bigpagenumber,		% circled page number on lower right
%topline=true,		% colored bar at the top of the frame 
%]{Zurich}

\usetheme{metropolis}

\usepackage{amsmath}
\usepackage{fontspec,xunicode,xltxtra}
\usepackage{pgf}
\usepackage{tikz}
\usetikzlibrary{patterns}
\usetikzlibrary{plotmarks}
\usetikzlibrary{arrows,decorations.pathmorphing,backgrounds,fit,positioning,shapes,chains}
\definecolor{yellow1}{rgb}{1,0.8,0.2}  
\usepackage{pgfplots}
\pgfplotsset{compat=newest}
\tikzset{elegant/.style={smooth,thick,samples=50,cyan}}
\tikzset{eaxis/.style={->,>=stealth}}

\let\oldequation=\equation
\let\oldendequation=\endequation
\renewenvironment{equation}{\oldequation\textstyle}{\oldendequation}
\setlength{\baselineskip}{0pt}
\renewcommand{\baselinestretch}{0pt}
\setlength{\partopsep}{0pt}
\setlength{\parsep}{0pt}
\setlength{\itemsep}{0cm}
\setlength{\topsep}{0cm}
\setlength{\parskip}{0pt}
\setlength{\lineskip}{0pt}
%\setitemize[1]{itemsep=0pt,partopsep=0pt,parsep=\parskip,topsep=0pt}
%%\usepackage[compatibility=false]{caption}
%%\captionsetup{font={scriptsize}}
\setbeamerfont{caption}{size=\tiny}
\setlength{\abovecaptionskip}{0pt}
\setlength{\belowcaptionskip}{0pt}
\setlength{\abovedisplayshortskip}{0pt}
\setlength{\belowdisplayshortskip}{0pt}
\setlength{\abovedisplayskip}{0pt}
\setlength{\belowdisplayskip}{0pt}
\setlength{\jot}{0pt}
\usepackage{setspace} 
\linespread{1.4}
\fi

\ifx\allfiles\undefined
\begin{document}
\fi
\section{Deep Learning Tricks}
% ----------------------------------------------------------------------------
\begin{frame}
\frametitle{Deep Learning Tricks}
	\small
	\begin{itemize}
		\item Derivative Checking
		\item Gradient Vanishing and Explosion
		\item Data Preparation
			\\\hspace{1cm} Mean-Subtraction and Normalization
			\\\hspace{1cm} PCA and Whitening 
			\\\hspace{1cm} Batch Normalization
		\item Weight Initialization
			\\\hspace{1cm}Pitfall: All Zero Initialization
			\\\hspace{1cm}Small Random Numbers
			\\\hspace{1cm}Calibrating the variances with 1/sqrt(n)
			\\\hspace{1cm}Sparse initialization
			\\\hspace{1cm}Xavier Initialization
		\item Dropout
	\end{itemize}
\end{frame}
% ----------------------------------------------------------------------------
\begin{frame}
\frametitle{Derivative Checking}
	\small
	\begin{itemize}
		\item Check analytical derivative by numerical version. \\
		$f^{\prime} (x)\approx \frac{f(x+h)-f(x-h)}{2h}$
		\item why not $\frac{f(x+h)-f(x)}{h}$ but $\frac{f(x+h)-f(x-h)}{2h}$?
		\\by Tylor expansion at $x$
		\\$\frac{f(x+h)-f(x)}{h}=\frac{1}{h}(f^{(0)}(x)+f^{(1)}(x)h+\frac{1}{2!}f^{(2)}(x)h^2
				+\sum_{n=3}^{\infty}\frac{1}{n!}f^{(n)}h^n-f(x))$
			\\\hspace{1.8cm}$=\frac{1}{h}(f^{(1)}(x)h+\sum_{n=2}^{\infty}\frac{1}{n!}f^{(n)}h^n)$
			\\\hspace{1.8cm}$=f^{(1)}(x)+\frac{1}{h}\sum_{n=2}^{\infty}\frac{1}{n!}f^{(n)}h^n)$
		\\$\frac{f(x+h)-f(x-h)}{2h}=\frac{1}{2h}(f^{(0)}(x)+f^{(1)}(x)h+\sum_{n=2}^{\infty}\frac{1}{n!}f^{(n)}h^n$ 
			\\\hspace{2cm}$-(f^{(0)}(x)+f^{(1)}(x)(-h)+\sum_{n=2}^{\infty}\frac{1}{n!}f^{(n)}(-h)^n))$
			\\\hspace{2cm}$=\frac{1}{2h}(f^{(1)}(x)2h+\sum_{n=2}^{\infty}\frac{2}{(2n-1)!}f^{(2n-1)}2h^{2n-1})))$
			\\\hspace{2cm}$=f^{(1)}(x)+\frac{1}{2h}\sum_{n=2}^{\infty}\frac{2}{(2n-1)!}f^{(2n-1)}2h^{2n-1})))$
			\\$(\frac{f(x+h)-f(x-h)}{2h}-f^{(1)}(x))\thicksim O(h^{2n-1}) \ll  (\frac{f(x+h)-f(x)}{h}-f^{(1)}(x))\thicksim O(h^n)$
	\end{itemize}
\end{frame}
% ----------------------------------------------------------------------------
\begin{frame}
\frametitle{Xavier Initialization}
	\small
	\begin{itemize}
		\item  In order to avoid neurons becoming too correlated and ending up in poor local minimina, it
is often helpful to randomly initialize parameters.One of the most frequent initializations used is called
Xavier initialization.
		\item Given a matrix $A$ of dimension $m * n$, select values $A_{ij}$ uniformly from $[−\epsilon,\epsilon]$, where
		\\\hspace{5cm}$\epsilon = \frac{\sqrt{6}}{\sqrt{m+n}}$
	\end{itemize}
\end{frame}
% ----------------------------------------------------------------------------
\ifx\allfiles\undefined
\end{document}
\fi
