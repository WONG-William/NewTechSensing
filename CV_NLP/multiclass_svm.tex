\ifx\allfiles\undefined
\documentclass[11 pt,t]{beamer}
\geometry{showframe,paperwidth=160mm,paperheight=120mm,margin=5mm,nohead,nofoot,nomarginpar}
%\usetheme[
%bullet=circle,		% Other option: square
%bigpagenumber,		% circled page number on lower right
%topline=true,		% colored bar at the top of the frame 
%]{Zurich}

\usetheme{metropolis}

\usepackage{amsmath}
\usepackage{fontspec,xunicode,xltxtra}
\usepackage{pgf}
\usepackage{tikz}
\usetikzlibrary{patterns}
\usetikzlibrary{plotmarks}
\usetikzlibrary{arrows,decorations.pathmorphing,backgrounds,fit,positioning,shapes,chains}
\definecolor{yellow1}{rgb}{1,0.8,0.2}  
\usepackage{pgfplots}
\pgfplotsset{compat=newest}
\tikzset{elegant/.style={smooth,thick,samples=50,cyan}}
\tikzset{eaxis/.style={->,>=stealth}}

\let\oldequation=\equation
\let\oldendequation=\endequation
\renewenvironment{equation}{\oldequation\textstyle}{\oldendequation}
\setlength{\baselineskip}{0pt}
\renewcommand{\baselinestretch}{0pt}
\setlength{\partopsep}{0pt}
\setlength{\parsep}{0pt}
\setlength{\itemsep}{0cm}
\setlength{\topsep}{0cm}
\setlength{\parskip}{0pt}
\setlength{\lineskip}{0pt}
%\setitemize[1]{itemsep=0pt,partopsep=0pt,parsep=\parskip,topsep=0pt}
%%\usepackage[compatibility=false]{caption}
%%\captionsetup{font={scriptsize}}
\setbeamerfont{caption}{size=\tiny}
\setlength{\abovecaptionskip}{0pt}
\setlength{\belowcaptionskip}{0pt}
\setlength{\abovedisplayshortskip}{0pt}
\setlength{\belowdisplayshortskip}{0pt}
\setlength{\abovedisplayskip}{0pt}
\setlength{\belowdisplayskip}{0pt}
\setlength{\jot}{0pt}
\usepackage{setspace} 
\linespread{1.4}
\fi

\ifx\allfiles\undefined
\begin{document}
\fi
\section{Multiclass SVM}
% ----------------------------------------------------------------------------
\begin{frame}
\frametitle{Linear Separable Binary SVM}
	\small
	\begin{itemize}
		\item hypothesis function:$h_{\theta}(x^{(i)})=\theta^Tx^{(i)}$\\
		\item predict function:
			\scriptsize
			\begin{equation*}
				y^{(i)}=
				\left\{
					\begin{aligned}
						1, if(h_{\theta}(x^{(i)}) > d_1)\\
						-1, if( h_{\theta}(x^{(i)}) < -d_2)
					\end{aligned}
				\right.
			\end{equation*}\\
		\item loss function: 
			$L(\theta)=||\theta||^2$ subject to $y^{(i)}\theta^Tx^{(i)} \geq \min(d_1,d_2)$\\
			(if two points $x^{(a)}$ and $x^{(b)}$ nearest to the line, loss function is geometric margin: 
			$\frac{|\theta^Tx^{(a)}|}{||\theta||}+\frac{|\theta^Tx^{(b)}|}{||\theta||}=\frac{d_1+d_2}{||\theta||}$, $d_1+d_2$ is constant, so loss function is simplified to $\max(\frac{1}{||\theta||})$, the same to $||\theta||^2$.)
		\item we can make $d_1=d_2=d$, and replace $\theta$ with $\frac{\theta}{d}$, the loss function is not changed, and predict function is simplified to:
			\scriptsize
			\begin{equation*}
				y^{(i)}=
				\left\{
					\begin{aligned}
						1, if(h_{\theta}(x^{(i)}) > 1)\\
						-1, if( h_{\theta}(x^{(i)}) < -1)
					\end{aligned}
				\right.
			\end{equation*}
		\item optimization: This is not unconstrained optimization, can't be solved by GD, but solved by Lagrange duality and SMO. 
	\end{itemize}
\end{frame}
% ----------------------------------------------------------------------------
\begin{frame}
\frametitle{Dual Problem Optimization}
	\small
	\begin{itemize}
		\item add Lagrange factor to construct function:
			\\ $L(\theta,\alpha)=\frac{1}{2}||\theta||^2-\sum_{i=1}^{m}\alpha_i(y^{(i)}\theta^Tx^{(i)}-1)$, subject to $\alpha_i \geq 0$
		\item Primal problem is: $\min_{\theta}\max_{\alpha}L(\theta,\alpha)$ 
		\item Dual problem is:$\max_{\alpha}\min_{\theta}L(\theta,\alpha)$
		\item to resolve dual problem, first resolve $\min_{\theta}L(\theta,\alpha)$ by the following way:\\
		$\nabla_{\theta}L(\theta,\alpha)=\theta-\sum_{i=1}^{m}\alpha_iy^{(i)}x^{(i)}=0$
		so, we get $\theta=\sum_{i=1}^{m}\alpha_iy^{(i)}x^{(i)}$
		\item The dual problem is to be:\\
		 	$\max_{\alpha}W(\alpha)=
				\frac{1}{2}\sum_{i,j=1}^{m}\alpha_i\alpha_jy^{(i)}y^{(j)}{x^{(i)}}^Tx^{(j)}
					-\sum_{i,j=1}^{m}\alpha_i\alpha_jy^{(i)}y^{(j)}{x^{(i)}}^Tx^{(j)}
					+\sum_{i=1}^{m}\alpha_i$
			$=\sum_{i=1}^{m}\alpha_i-\frac{1}{2}\sum_{i,j=1}^{m}\alpha_i\alpha_jy^{(i)}y^{(j)}{x^{(i)}}^Tx^{(j)}$, subject to $\alpha_i \geq 0$.
		\item This can be resolved by GD.
	\end{itemize}
\end{frame}
% ----------------------------------------------------------------------------
\begin{frame}
\frametitle{Nonlinear Separable Binary SVM: Kernel Trick}
	\small
	\begin{itemize}
		\item The dual problem can be written into inner product manner:\\
		 	$\max_{\alpha}W(\alpha)
					=\sum_{i=1}^{m}\alpha
					-\frac{1}{2}\sum_{i,j=1}^{m}\alpha_i\alpha_jy^{(i)}y^{(j)}{x^{(i)}}^Tx^{(j)}$
					, subject to $\alpha_i \geq 0$.\\
					$=\sum_{i=1}^{m}\alpha
					 -\frac{1}{2}\sum_{i,j=1}^{m}\alpha_i\alpha_jy^{(i)}y^{(j)}<x^{(i)},x^{(j)}>$
					, subject to $\alpha_i \geq 0$.\\
		\item hypothesis function:$h_{\theta}(x^{(i)})=\theta^Tx^{(i)}$ can't separate training samples, i.e. nonlinear separable, we can consider more complex relationship between $y$ and $x$. Such as:
				$h_{\theta}(x^{(i)})=\sum_{a,b=0}^n\theta_{a,b}x^{(i)}_ax^{(i)}_b=\theta^Tx_{trans}^{(i)}$,
				where $n$ is dimension of $X$. 
		\item in loss function: $<x_{trans}^{(i)},x_{trans}^{(j)}>=\sum_{a,b=0}^n(x^{(i)}_a)^2(x^{(i)}_b)^2=((x^{(i)})^T(x^{(i)}))^2$
		\item $<x_{trans}^{(i)},x_{trans}^{(j)}>$ is calculated in time cost of $O(n)$ instead of $O(n^2)$.
		\item This is defined as polynomial kernel: $<x_{trans}^{(i)},x_{trans}^{(j)}>=((x^{(i)})^T(x^{(j)})+c)^d$, which mapping original linear $y$ on $x$ to polynomial $y$ on $x$.

	\end{itemize}
\end{frame}
% ----------------------------------------------------------------------------
\begin{frame}
\frametitle{Nonlinear Separable Binary SVM: Kernel Trick}
	\small
	\begin{itemize}
		\item other kernel function:
		\item Gaussian Kernel (radial basis function)
			\\$<x_{trans}^{(i)},x_{trans}^{(j)}>=\exp -\frac{||x^{(i)}-x^{(j)}||^2}{2\sigma^2}$
			\\which map to unlimited polynomial $y$ on $x$ by the idea of following function:
			\\Tylor expansion $e^x=1+\frac{x}{1}+\frac{x^2}{2!}+\frac{x^3}{3!}+...,-\infty<x<\infty$
		\item sigmoid: $<x_{trans}^{(i)},x_{trans}^{(j)}>=\tanh ((x^{(i)})^Tx^{(j)}+c)$
	\end{itemize}
\end{frame}
% ----------------------------------------------------------------------------
\begin{frame}
\frametitle{Nonlinear Separable Binary SVM: Soft Margin}
	\small
	\begin{itemize}
		\item hypothesis function:$h_{\theta}(x^{(i)})=\theta^Tx^{(i)}$\\
		\item predict function:
			\begin{equation*}
				y^{(i)}=
				\left\{
					\begin{aligned}
						1, if(h_{\theta}(x^{(i)}) > 1)\\
						-1, if( h_{\theta}(x^{(i)}) < -1)
					\end{aligned}
				\right.
			\end{equation*}\\
		\item loss function: 
			$l(\theta)=\lambda||\theta||^2+\frac{1}{n}\sum_{i=1}^n\max(0,1-y^{(i)}\theta^tx^{(i)})$ 
		\item derivatives:
				\begin{equation*}
					\nabla_{\theta_j} L=
					\left\{
						\begin{aligned}
							1(
								1
								-y^{(i)}\theta^tx^{(i)}
								 > 0
							)x^{(i)}\\	
							1(
								1
								-y^{(i)}\theta^tx^{(i)}
								 < 0
							)0
						\end{aligned}
					\right.
				\end{equation*}
		\item sometimes, to enlarge the margin we have the following loss function:\\
			$l(\theta)=\lambda||\theta||^2+\frac{1}{n}\sum_{i=1}^n\max(0,1-y^{(i)}\theta^tx^{(i)}+\Delta)$, where $\Delta>0$
	\end{itemize}
\end{frame}
% ----------------------------------------------------------------------------
\begin{frame}
\frametitle{Multiclass SVM}
	\small
	\begin{itemize}
		\item binary classification: $L_i=max(0,1-y^{(i)}\theta^Tx^{(i)})=max(0,1-m_i)$
		\\loss function: $L=L_{data}+\lambda L_{norm}=\frac{1}{m}\sum_{i=1}^{m}max(0,1-y^{(i)}\theta^Tx^{(i)})+\lambda L_{norm}$
		\item multiple classification:$L_i=\sum_{j\neq y^{(i)}}^{k}
										max(0,
											\theta_j^Tx^{(i)}-
											\theta_{y^{(i)}}^Tx^{(i)}+
											\Delta)$
		\\\hspace{3cm}or $L_i=			max(0,
											max_{j\neq y^{(i)}}^{k}(
											\theta_j^Tx^{(i)})-
											\theta_{y^{(i)}}^Tx^{(i)}+
											1)$
		\item derivative\\
\footnotesize
				\begin{equation}
					\nabla_{\theta_j} L=
					\left\{
						\begin{aligned}
							1(
								\theta_j^Tx^{(i)}
								-\theta_{y^{(i)}}^Tx^{(i)}
								+\Delta > 0
							)x^{(i)}, if(j\neq y^{(i)})\\
							-(
								\sum_{j\neq y^{(i)}}^k
									1(
										\theta_j^Tx^{(i)}
										-\theta_{y^{(i)}}^Tx^{(i)}
										+\Delta > 0
									)
							)x^{(i)}, if(j= y^{(i)})
						\end{aligned}
					\right.
				\end{equation}
				\begin{equation}
					\nabla_{\theta_j} L=
					\left\{
						\begin{aligned}
							1(
										\theta_j^Tx^{(i)}-
										\theta_{y^{(i)}}^Tx^{(i)}+
										1 > 0
							)x^{(i)}, 
								if(j\neq y^{(i)} and \theta_j^Tx^{(i)} is max)\\
							0, if(j\neq y^{(i)} and \theta_j^Tx^{(i)} is not max)\\
							-(
								\sum_{j\neq y^{(i)}}^k
									1(
											\theta_j^Tx^{(i)}-
											\theta_{y^{(i)}}^Tx^{(i)}+
											1 > 0
									)
							)x^{(i)}, if(j= y^{(i)})
						\end{aligned}
					\right.
				\end{equation}
	\end{itemize}
\end{frame}
% ----------------------------------------------------------------------------
\ifx\allfiles\undefined
\end{document}
\fi
