\ifx\allfiles\undefined
\documentclass[11 pt,t]{beamer}
\geometry{showframe,paperwidth=160mm,paperheight=120mm,margin=5mm,nohead,nofoot,nomarginpar}
%\usetheme[
%bullet=circle,		% Other option: square
%bigpagenumber,		% circled page number on lower right
%topline=true,		% colored bar at the top of the frame 
%]{Zurich}

\usetheme{metropolis}

\usepackage{amsmath}
\usepackage{fontspec,xunicode,xltxtra}
\usepackage{pgf}
\usepackage{tikz}
\usetikzlibrary{patterns}
\usetikzlibrary{plotmarks}
\usetikzlibrary{arrows,decorations.pathmorphing,backgrounds,fit,positioning,shapes,chains}
\definecolor{yellow1}{rgb}{1,0.8,0.2}  
\usepackage{pgfplots}
\pgfplotsset{compat=newest}
\tikzset{elegant/.style={smooth,thick,samples=50,cyan}}
\tikzset{eaxis/.style={->,>=stealth}}

\let\oldequation=\equation
\let\oldendequation=\endequation
\renewenvironment{equation}{\oldequation\textstyle}{\oldendequation}
\setlength{\baselineskip}{0pt}
\renewcommand{\baselinestretch}{0pt}
\setlength{\partopsep}{0pt}
\setlength{\parsep}{0pt}
\setlength{\itemsep}{0cm}
\setlength{\topsep}{0cm}
\setlength{\parskip}{0pt}
\setlength{\lineskip}{0pt}
%\setitemize[1]{itemsep=0pt,partopsep=0pt,parsep=\parskip,topsep=0pt}
%%\usepackage[compatibility=false]{caption}
%%\captionsetup{font={scriptsize}}
\setbeamerfont{caption}{size=\tiny}
\setlength{\abovecaptionskip}{0pt}
\setlength{\belowcaptionskip}{0pt}
\setlength{\abovedisplayshortskip}{0pt}
\setlength{\belowdisplayshortskip}{0pt}
\setlength{\abovedisplayskip}{0pt}
\setlength{\belowdisplayskip}{0pt}
\setlength{\jot}{0pt}
\usepackage{setspace} 
\linespread{1.4}
\fi

\ifx\allfiles\undefined
\begin{document}
\fi
\section{Neural Network}
% ----------------------------------------------------------------------------
\begin{frame}
\frametitle{Neural Network}
	\small
	\begin{itemize}
		\item Activation Function 
		\item Forward Propagation
		\item Backward Propagation
		\item Loss Function
	\end{itemize}
\end{frame}
% ----------------------------------------------------------------------------
\begin{frame}
\frametitle{Activation Function}
	\small
	\begin{itemize}
		\item neuron is a function, mapping from linear combination of multiple vector $x$ to a vector $y$ by some type of activation function $f$. 
			$y=f(\theta^Tx)$ 
		\item there are multiple types of activation functions:
			\\sigmoid function: $f(y)=\frac{1}{1+e^{-y}}$, $f^\prime(y)=f(y)(1-f(y))$
			\\tangent function: $f(y)=\frac{e^{y}-e^{-y}}{e^{y}+e^{-y}}$, $f^\prime(y)=1-(f(y))^2$
			\\rectified linear unit (ReLU): $f(y)=\max(0,y)$, $f^\prime(y)=1(x>0)$
			\\maxout: $f(y)=\max_{i=0}^{k}(y_i)$, $f^\prime(y)=1(y_i>0)$
		\item multiple neurons connected from layer to layer, is neural network. 
			\footnotesize
			\\\hspace{1cm}Inside layer, neurons are not connected to each other. 
			\\\hspace{1cm}Between adjacent layers, there are connections between neurons. 
			\\\hspace{1cm}Between nonadjacent layers, there are not connections between neurons.
		\item input layer doesn't have activation function, every hidden layer has an activation function, and output layer has only loss function.
	\end{itemize}
\end{frame}
% ----------------------------------------------------------------------------
\begin{frame}
\frametitle{Forward Propagation}
	\small
	\begin{itemize}
		\item define 
			\\\hspace{1cm}previous layer $y_{prev}=f_{prev}(\theta_{prev}^Tx_{prev})$,
			\\\hspace{1cm}now layer $y_{now}=f_{now}(\theta_{now}^Tx_{now})$,
			\\\hspace{1cm}next layer $y_{next}=f_{next}(\theta_{next}^Tx_{next})$
		\item forward propagation is calculate from previous layer to now layer, and to next layer. In this way, the result of final layer $y_{final}$ is calculated.
			\\\hspace{1cm}$x_{now}=y_{prev}$
			\\\hspace{1cm}$x_{next}=y_{now}$
	\end{itemize}
\end{frame}
% ----------------------------------------------------------------------------
\begin{frame}
\frametitle{Backward Propagation}
	\small
	\begin{itemize}
		\item chain rule of derivative:
			\\$\delta_{now}=\frac{dy_{final}}{d(\theta_{now}x_{now})}$
			\\	$=\frac{dy_{final}}{d(\theta_{next}x_{next})}\frac{d(\theta_{next}x_{next})}{d(\theta_{now}x_{now})}
				=\frac{dy_{final}}{d(\theta_{next}x_{next})}\frac{d(\theta_{next}y_{now})}{d(\theta_{now}x_{now})}
				=\frac{dy_{final}}{d(\theta_{next}x_{next})}\frac{\theta_{next}dy_{now}}{d(\theta_{now}x_{now})}$
			\\$	=\frac{dy_{final}}{d(\theta_{next}x_{next})}\frac{\theta_{next}df_{now}(\theta_{now}x_{now})}{d(\theta_{now}x_{now})}
				=\frac{dy_{final}}{d(\theta_{next}x_{next})}\frac{\theta_{next}f^\prime_{now}d(\theta_{now}x_{now})}{d(\theta_{now}x_{now})}$
			\\$ =\frac{dy_{final}}{d(\theta_{next}x_{next})}\theta_{next}f^\prime_{now}$
			\\$=\delta_{next}\theta_{next}f^\prime_{now}$
		\item derivatives:
			\\$\frac{dy_{final}}{d(\theta_{now})}
				=\frac{dy_{final}}{d(\theta_{now}x_{now})}\frac{d(\theta_{now}x_{now})}{d\theta_{now}}
				=\frac{dy_{final}}{d(\theta_{now}x_{now})}x_{now}$
		\item by chain rule, we can calculate $\frac{dy_{final}}{d(\theta_{now}x_{now})}$ in backward way. And calculate $\frac{dy_{final}}{d(\theta_{now})}$ in every layer. And use loss function, GD to get optimal $\theta$ for every layer.
	\end{itemize}
\end{frame}
% ----------------------------------------------------------------------------
\begin{frame}
\frametitle{Loss Function}
	\small
	\begin{itemize}
		\item Data Loss\\
			\hspace{1cm}0-1 loss\\
			\hspace{1cm}log loss\\
			\hspace{1cm}hinge loss\\
			\hspace{1cm}squared loss\\
			\hspace{1cm}exponential loss
		\item Regulation Loss\\
			\hspace{1cm}$L_1$ norm\\
			\hspace{1cm}$L_2$ norm
	\end{itemize}
\end{frame}
% ----------------------------------------------------------------------------
\begin{frame}
\frametitle{Problem Definition}
	\small
	\begin{itemize}
		\item The loss function is used to evaluate the learning target, by deriving derivative and control iterative processes. \\
			\hspace{1cm}$L=L_{data}+L_{norm}$\\
			\hspace{1.5cm}$=\frac{1}{N}\sum_{i=0}^N L(y^{(i)},h_{\theta}(x^{(i)})) 
						+ \lambda L_R$
		\item $L_{data}$ is data loss, to evaluate the classification.
		\item $L_{regualtion}$ is regulation loss, to evaluate the complexity of model to avoiding over-fit.
		\item actually $L$ has uniform form. Let $m_i=y^{(i)}\theta^Tx^{(i)}$, $L_i=L(m_i)$.\\
			\hspace{1cm}$L_{data}=\frac{1}{N}\sum_{i=0}^N L(m_i)$
		\item $h_{\theta^T}(x^{(i)})$ is called score function, comparing to hypothesis function of ML.\\
			\hspace{1cm} $L$ is called loss function, comparing to target function of ML.\\
			\hspace{1cm} Derivative is the same to ML.
	\end{itemize}
\end{frame}
% ----------------------------------------------------------------------------
\begin{frame}
\frametitle{Data Loss: 0-1 Loss}
	\small
	\begin{itemize}
		\item 0-1 loss: if $y^{(i)}$ and $\theta^Tx^{(i)}$ have the same sign, $L_i=0$.\\
			\hspace{1cm}if they have different sign, $L_i=1$, totally in the following equation:\\
				\begin{equation}
					L_{01}=
					\left\{
						\begin{aligned}
							0, if(m\geq 0)\\
							1, if(m < 0)
						\end{aligned}
					\right.
				\end{equation}
		\item Derivative is not analytic, so it is seldom used.
	\end{itemize}
\end{frame}
% ----------------------------------------------------------------------------
\begin{frame}
\frametitle{Data Loss: Log Loss}
	\small
	\begin{itemize}
		\item log loss
			$L_i=y^{(i)}\log h_{\theta}(x)+{(1-y^{(i)})}\log (1-h_{\theta}(x^{(i)}))$\\
				\hspace{1cm}$=y^{(i)}\log \frac{1}{1+e^{-\theta ^T x}}+{(1-y^{(i)})}\log (1-\frac{1}{1+e^{-\theta ^T x}})$\\
				\hspace{1cm}$=y^{(i)}\log \frac{1}{1+e^{-\theta ^T x}}+{(1-y^{(i)})}\log (\frac{1}{1+e^{\theta ^T x}})$\\
				\hspace{1cm}$=\log \frac{1}{1+e^{-\widetilde{y^{(i)}}\theta ^T x}}$
				\hspace{1cm} (when $y^{(i)}=0, \widetilde{y^{(i)}}=-1$; $y^{(i)}=1, \widetilde{y^{(i)}}=1$)	\\
				\hspace{1cm}$=\log \frac{1}{1+e^{-m_i}}$\\
				\hspace{1cm} softmax and logistic have the same expression.
		\item cross entropy to data loss: $H(p,q)=-\sum_x p(x)\log q(x)$.\\
			softmax	log loss function: 
			$L_{data}=-\frac{1}{m}\sum_{i=1}^{m}\sum_{j=1}^{k}1\{y_j^{(i)}=1\}\log({\frac{e^{\theta_j^Tx^{(i)}}}{\sum_{l=1}^{k} e^{\theta_l ^T x^{(i)}}}})$, with $p(x)=1\{y_j=1\}$ \\
		\item derivative\\
		\hspace{0cm}$=-\frac{1}{m}\sum_{i=1}^{m}
						(
							{1\{y_j^{(i)}=1\}}
								-\log (p(y^{(i)}=j|x^{(i)},\theta))
						){x^{(i)}}$\\
	\end{itemize}
\end{frame}
% ----------------------------------------------------------------------------
\begin{frame}
\frametitle{Data Loss: Hinge Loss}
	\small
	\begin{itemize}
		\item binary classification: $L_i=max(0,1-y^{(i)}\theta^Tx^{(i)})=max(0,1-m_i)$
		\\loss function: $L=L_{data}+\lambda L_{norm}=\frac{1}{m}\sum_{i=1}^{m}max(0,1-y^{(i)}\theta^Tx^{(i)})+\lambda L_{norm}$
		\item multiple classification:$L_i=\sum_{j\neq y^{(i)}}^{k}
										max(0,
											\theta_j^Tx^{(i)}-
											\theta_{y^{(i)}}^Tx^{(i)}+
											\Delta)$
		\\\hspace{3cm}or $L_i=			max(0,
											max_{j\neq y^{(i)}}^{k}(
											\theta_j^Tx^{(i)})-
											\theta_{y^{(i)}}^Tx^{(i)}+
											1)$
		\item derivative\\
\footnotesize
				\begin{equation}
					\nabla_{\theta_j} L=
					\left\{
						\begin{aligned}
							1(
								\theta_j^Tx^{(i)}
								-\theta_{y^{(i)}}^Tx^{(i)}
								+\Delta > 0
							)x^{(i)}, if(j\neq y^{(i)})\\
							-(
								\sum_{j\neq y^{(i)}}^k
									1(
										\theta_j^Tx^{(i)}
										-\theta_{y^{(i)}}^Tx^{(i)}
										+\Delta > 0
									)
							)x^{(i)}, if(j= y^{(i)})
						\end{aligned}
					\right.
				\end{equation}
				\begin{equation}
					\nabla_{\theta_j} L=
					\left\{
						\begin{aligned}
							1(
										\theta_j^Tx^{(i)}-
										\theta_{y^{(i)}}^Tx^{(i)}+
										1 > 0
							)x^{(i)}, 
								if(j\neq y^{(i)} and \theta_j^Tx^{(i)} is max)\\
							0, if(j\neq y^{(i)} and \theta_j^Tx^{(i)} is not max)\\
							-(
								\sum_{j\neq y^{(i)}}^k
									1(
											\theta_j^Tx^{(i)}-
											\theta_{y^{(i)}}^Tx^{(i)}+
											1 > 0
									)
							)x^{(i)}, if(j= y^{(i)})
						\end{aligned}
					\right.
				\end{equation}
	\end{itemize}
\end{frame}
% ----------------------------------------------------------------------------
\begin{frame}
\frametitle{Data Loss: Squared Loss and Exponential Loss}
	\small
	\begin{itemize}
		\item squared loss is for linear regression, $L_i=(y^{(i)}-\theta^Tx^{(i)})^2$
		\hspace{1cm} derivative:$\nabla_{\theta_j} L=x^{(i)}$
		\item exponential loss for boosting algorithm, $L_i=exp(-y^{(i)}\theta^Tx^{(i)})=exp(-m_i)$
		\hspace{1cm} derivative:
		\item all loss function is decreased by increasing x, except squared loss and 0-1 loss.
	\end{itemize}
\end{frame}
% ----------------------------------------------------------------------------
\begin{frame}
\frametitle{Regulation Loss}
	\small
	\begin{itemize}
	\item $L_{norm}=L_p=(\sum_i|\theta_i|^p)^{\frac{1}{p}}$
	\item $L_1=\sum_i|\theta_i|$
	\item $L_2=(\sum_i|\theta_i|^2)^{\frac{1}{2}}$, which is the most widely used.
	\item $L_{norm}$ should be added into the calculation of derivatives.
	\end{itemize}
\end{frame}
% ----------------------------------------------------------------------------
\ifx\allfiles\undefined
\end{document}
\fi
